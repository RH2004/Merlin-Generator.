\title{Introduction to Calculus}

\slide{Limits}{
  \definition{Limit}{A value that a function approaches as the input approaches some value.}
  
  \equation[animate=reveal]{\lim_{x \to a} f(x) = L}
  
  \note{Emphasize that limits are fundamental to calculus and form the basis for derivatives and integrals.}
}

\slide{L'Hôpital's Rule}{
  \theorem{L'Hôpital's Rule}{If the limit of f(x)/g(x) as x approaches a gives an indeterminate form (0/0 or ∞/∞), then the limit equals lim f'(x)/g'(x) if it exists.}
  
  \equation{\lim_{x \to a} \frac{f(x)}{g(x)} = \lim_{x \to a} \frac{f'(x)}{g'(x)}}
  
  \proof{Follows from the Cauchy Mean Value Theorem. The proof requires showing that the ratio of function values approaches the ratio of their derivatives under certain continuity and differentiability conditions.}
}

\slide{Derivative Applications}{
  \algorithm{Finding Critical Points}{
    Set f'(x) = 0,
    Solve for x to get critical points,
    Test each point using second derivative test,
    f''(x) > 0 implies local minimum,
    f''(x) < 0 implies local maximum,
    f''(x) = 0 requires further analysis
  }
  
  \note{Walk through an example on the board using a simple polynomial function.}
}

\slide{Fundamental Theorem of Calculus}{
  \theorem{Fundamental Theorem (Part 1)}{If f is continuous on [a,b] and F is an antiderivative of f, then the definite integral from a to b of f(x)dx equals F(b) - F(a).}
  
  \equation[animate=highlight]{\int_a^b f(x)\,dx = F(b) - F(a)}
  
  This connects differentiation and integration as inverse operations.
}

\slide{Integration Techniques}{
  \algorithm{Integration by Parts}{
    Choose u and dv from the integrand,
    Compute du and v,
    Apply formula: ∫u dv = uv - ∫v du,
    Simplify the resulting integral,
    Repeat if necessary
  }
  
  \equation{\int u\,dv = uv - \int v\,du}
}
